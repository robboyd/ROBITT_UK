\documentclass[]{article}
\usepackage{lmodern}
\usepackage{amssymb,amsmath}
\usepackage{ifxetex,ifluatex}
\usepackage{fixltx2e} % provides \textsubscript
\ifnum 0\ifxetex 1\fi\ifluatex 1\fi=0 % if pdftex
  \usepackage[T1]{fontenc}
  \usepackage[utf8]{inputenc}
\else % if luatex or xelatex
  \ifxetex
    \usepackage{mathspec}
  \else
    \usepackage{fontspec}
  \fi
  \defaultfontfeatures{Ligatures=TeX,Scale=MatchLowercase}
\fi
% use upquote if available, for straight quotes in verbatim environments
\IfFileExists{upquote.sty}{\usepackage{upquote}}{}
% use microtype if available
\IfFileExists{microtype.sty}{%
\usepackage{microtype}
\UseMicrotypeSet[protrusion]{basicmath} % disable protrusion for tt fonts
}{}
\usepackage[margin=1in]{geometry}
\usepackage[unicode=true]{hyperref}
\hypersetup{
            pdftitle={ROBITT\_template},
            pdfauthor={Rob Boyd},
            pdfborder={0 0 0},
            breaklinks=true}
\urlstyle{same}  % don't use monospace font for urls
\usepackage{longtable,booktabs}
\usepackage{graphicx,grffile}
\makeatletter
\def\maxwidth{\ifdim\Gin@nat@width>\linewidth\linewidth\else\Gin@nat@width\fi}
\def\maxheight{\ifdim\Gin@nat@height>\textheight\textheight\else\Gin@nat@height\fi}
\makeatother
% Scale images if necessary, so that they will not overflow the page
% margins by default, and it is still possible to overwrite the defaults
% using explicit options in \includegraphics[width, height, ...]{}
\setkeys{Gin}{width=\maxwidth,height=\maxheight,keepaspectratio}
\IfFileExists{parskip.sty}{%
\usepackage{parskip}
}{% else
\setlength{\parindent}{0pt}
\setlength{\parskip}{6pt plus 2pt minus 1pt}
}
\setlength{\emergencystretch}{3em}  % prevent overfull lines
\providecommand{\tightlist}{%
  \setlength{\itemsep}{0pt}\setlength{\parskip}{0pt}}
\setcounter{secnumdepth}{0}
% Redefines (sub)paragraphs to behave more like sections
\ifx\paragraph\undefined\else
\let\oldparagraph\paragraph
\renewcommand{\paragraph}[1]{\oldparagraph{#1}\mbox{}}
\fi
\ifx\subparagraph\undefined\else
\let\oldsubparagraph\subparagraph
\renewcommand{\subparagraph}[1]{\oldsubparagraph{#1}\mbox{}}
\fi

\title{ROBITT\_template}
\author{Rob Boyd}
\date{18 May 2022}

\begin{document}
\maketitle

\section{1. Iteration}\label{iteration}

\subsubsection{1.1 ROBITT iteration
number}\label{robitt-iteration-number}

\begin{longtable}[]{@{}lll@{}}
\toprule
Iteration & Comments &\tabularnewline
\midrule
\endhead
& &\tabularnewline
\bottomrule
\end{longtable}

\section{2. Research statement and pre-bias
assessments}\label{research-statement-and-pre-bias-assessments}

\subsection{Statistical population of
interest}\label{statistical-population-of-interest}

\textbf{2.1 Define the statistical target population about which you
intend to make inferences.}

\begin{longtable}[]{@{}lll@{}}
\toprule
\begin{minipage}[b]{0.05\columnwidth}\raggedright\strut
Domain\strut
\end{minipage} & \begin{minipage}[b]{0.05\columnwidth}\raggedright\strut
Extent\strut
\end{minipage} & \begin{minipage}[b]{0.05\columnwidth}\raggedright\strut
Resolution\strut
\end{minipage}\tabularnewline
\midrule
\endhead
\begin{minipage}[t]{0.05\columnwidth}\raggedright\strut
Geographic\strut
\end{minipage} & \begin{minipage}[t]{0.05\columnwidth}\raggedright\strut
\strut
\end{minipage} & \begin{minipage}[t]{0.05\columnwidth}\raggedright\strut
\strut
\end{minipage}\tabularnewline
\begin{minipage}[t]{0.05\columnwidth}\raggedright\strut
Temporal\strut
\end{minipage} & \begin{minipage}[t]{0.05\columnwidth}\raggedright\strut
\strut
\end{minipage} & \begin{minipage}[t]{0.05\columnwidth}\raggedright\strut
\strut
\end{minipage}\tabularnewline
\begin{minipage}[t]{0.05\columnwidth}\raggedright\strut
Taxonomic (or other relevant organismal domain such as functional group
etc.)\strut
\end{minipage} & \begin{minipage}[t]{0.05\columnwidth}\raggedright\strut
\strut
\end{minipage} & \begin{minipage}[t]{0.05\columnwidth}\raggedright\strut
\strut
\end{minipage}\tabularnewline
\begin{minipage}[t]{0.05\columnwidth}\raggedright\strut
Environmental\strut
\end{minipage} & \begin{minipage}[t]{0.05\columnwidth}\raggedright\strut
\strut
\end{minipage} & \begin{minipage}[t]{0.05\columnwidth}\raggedright\strut
\strut
\end{minipage}\tabularnewline
\bottomrule
\end{longtable}

\section{Inferential goals}\label{inferential-goals}

\textbf{2.2 What are your inferential goals?}

\subsection{Data provenance}\label{data-provenance}

\textbf{2.3 From where were your data acquired (please provide
citations, including a DOI, wherever possible)? What are their key
features in respect of the inferential aims of your study (see the
guidance document for examples)?}

\subsection{Data processing}\label{data-processing}

\textbf{2.4 Provide details of, and the justification for, all of the
steps that you have taken to clean the data described above prior to
analyses.}

\section{3. Bias assessment and
mitigation}\label{bias-assessment-and-mitigation}

\subsection{Assessment resolutions}\label{assessment-resolutions}

\textbf{3.1 At what geographic, temporal and taxonomic resolutions
(i.e.~scales or grain sizes) will you conduct your bias assessment?}

\subsection{Geographic domain}\label{geographic-domain}

\textbf{3.2 Are the data sampled from a representative portion of
geographical space in the domain of interest?}

\textbf{3.3 Are your data sampled from the same portions of geographic
space across time periods?}

\textbf{3.4 If the answers to the above questions revealed any potential
geographic biases, or temporal variation in geographic coverage, please
explain, in detail, how you plan to mitigate them.}

\subsection{Environmental domain}\label{environmental-domain}

\textbf{3.5 Are your data sampled from a representative portion of
environmental space in the domain of interest?}

\textbf{3.6 Are your data sampled from the same portion of environmental
space across time periods?}

\textbf{3.7 If the answers to the above questions revealed any potential
environmental biases, or temporal variation in environmental coverage,
please explain, in detail, how you plan to mitigate them.}

\subsection{Taxonomic domain (or other organismal domain, e.g.,
phylogenetic, trait space
etc.)}\label{taxonomic-domain-or-other-organismal-domain-e.g.-phylogenetic-trait-space-etc.}

\textbf{Is the sampled portion of the taxonomic (or phylogenetic, trait
or other space if more relevant) space representative of the taxonomic
(or other) domain of interest?}

\textbf{3.9 Do your data pertain to the same taxa/taxonomic domain
across time periods?}

\textbf{3.10 If the answers to the above questions revealed any
potential taxonomic biases, or temporal variation in taxonomic coverage,
please explain, in detail, how you plan to mitigate them.}

\subsection{Other potential biases}\label{other-potential-biases}

\textbf{3.11 Are there other potential temporal biases in your data that
relate to variables other than ecological states?}

\textbf{3.12 Are you aware of any other potential biases not covered by
the above questions that might cause problems for your inferences?}

\textbf{3.13 If questions 3.11 or 3.12 revealed any important potential
biases, please explain how you will mitigate them.}

\section{4. Supporting references}\label{supporting-references}

\end{document}
